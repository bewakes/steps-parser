\documentclass{beamer}

\usefonttheme{serif} % default family is serif
\usepackage{fontspec}
\usepackage{braket}
\usepackage{amsmath}
\usepackage{amssymb}


\title{Dependency Parser for Nepali: Methodology Elaboration}
\author{Bibek Pandey}
\institute{076MSCSK003}
\date{June, 2022}

\newcommand{\V}[1]{\textbf{#1}}
\newcommand{\nn}{\\~\\}

\begin{document}

\begin{frame}
\titlepage
\end{frame}

\begin{frame}
\frametitle{}
\huge\textbf{Multilingual Embeddings}
\end{frame}

\begin{frame}
\frametitle{General Idea}
\begin{itemize}
\item Predefined fastText embeddings for English($E^{en}$), Hindi($E^{hi}$) and Nepali($E^{np}$).
\item Create multilingual embeddings using a method called MultiCCA.
\item Use English embedding space as multilingual embeddings space.
\item For each pair with English, create two projection matrices to project to common space for each pair: $T_{hi \rightarrow hi,en}, T_{en \rightarrow hi,en}$ and $T_{np \rightarrow np,en}, T_{en \rightarrow np,en}$
\item Define multilingual embeddings as:
    $$E_{CCA}(np, word) = T_{en \rightarrow np,en}^{-1}T_{np \rightarrow np,en}E^{np}(word)$$
\end{itemize}
\end{frame}

\begin{frame}
\frametitle{Cannonical Correlation Analysis(CCA)}
Let $\Sigma \in \mathbb{R}^{n_1 \times d_1}$ and $\Omega \in \mathbb{R}^{n_2 \times d_2}$ be vector space embeddings of two different vocabularies.
\nn
Let $\Sigma' \subset \Sigma$ and $\Omega' \subset \Omega$ be the embeddings which have some mappings either using bilingual dictionary or using a parallel corpus.
\nn
Let $x \in \Sigma'$ and $y \in \Omega'$ be two vectors and, $v$ and $w$ be two projection directions that project embeddings from each language to a common space.
\nn
Now, projected vectors are $x' = xv, y' = yw$. \\
Correlation between projected vectors
$$\rho(x', y') = \frac{E[x'y']}{\sqrt{E[x'^2]E[y'^2]}} $$
\end{frame}

\begin{frame}
\frametitle{Cannonical Corrleation Analysis(CCA)}
Then, the required projection matrices $v$ and $w$ are given as,
\begin{equation}
\begin{aligned}
(v, w) &= CCA(x, y) \\
&= \underset{v, w}{argmax} \; \rho(xv, yw)
\end{aligned}
\end{equation}
\end{frame}

\begin{frame}
\frametitle{}
\huge\textbf{Multilingual Brown Clusters}
\end{frame}

\begin{frame}
\frametitle{Multilingual Brown Clusters}
\begin{itemize}
    \item Assigns unique ID to each words group based on their co-occurences in corpus.
    \item First create English brown clusters from Open American National Corpus(11 million words)
    \item Using the Multilingual embeddings, for each cluster average the embedding vectors to obtain vector representation for each cluster.
    \item For Nepali and Hindi, use the multilingual embeddings and cosine similarity to obtain corresponding cluster vector.
    \item For OOV(Out of Vocabulary) words, the cluster representation is the average of all cluster vectors.
    \item English Corpus Source: https://anc.org/data/oanc/download/
\end{itemize}
\end{frame}

\begin{frame}
\frametitle{}
{\huge\textbf{Language Embeddings}}
\nn
These are just one-hot encodings of the languages. The dimension is 3x1.
\end{frame}

\begin{frame}
\frametitle{}
{\huge\textbf{Token Embeddings}}
\nn
The multilingual embeddings, Brown cluster embeddings and language embeddings are concatenated and fed into the neural network.
\end{frame}

\begin{frame}
\frametitle{}
\huge\textbf{BiLSTM Representation}
\end{frame}

\begin{frame}
\frametitle{}
\huge\textbf{MLPs as Loss Functions}
\end{frame}


\begin{frame}
\frametitle{Novelty}
\begin{itemize}
    \item Use of multilingual BERT embeddings along with language embeddings and brown clusters for token representation.
    \item Using that token representation with BiLSTM-based architecture proposed by Kiperwasser(2016).
\end{itemize}
\end{frame}

\begin{frame}
\frametitle{}
\huge\textbf{Justifications}
\end{frame}

\begin{frame}
\frametitle{Justifications}
\begin{itemize}
    \item \textbf{Transition Based System}: When contextualized embeddings are used, accuracy of both graph-based and transition-based parsers is virtually equivalent.[Kulmizev et al. 2019].
    \item \textbf{BERT embeddings}: It is a contextualized embedding which has been shown to be effective in various NLP tasks and contextualized embeddings are almost a standard now. Also studies have shown that implicit syntactic information is captured in BERT[Manning 2019].
    \item \textbf{Language embeddings}: With a hope to capture directional relations in sentences of different languages.
    \item \textbf{Brown Clusters}:
    \item \textbf{BiLSTM}:
    \item \textbf{MLP}: To have a non-linear function that evaluates the performance.
\end{itemize}
\end{frame}

\begin{frame}
\frametitle{}
\huge\textbf{References}
\end{frame}


\begin{frame}
\frametitle{References}
\begin{itemize}
    \item Improving Vector Space Word Representations Using Multilingual Correlation. \url{https://www.manaalfaruqui.com/papers/eacl14-vectors.pdf}
    \item Deep Contextualized Word Embeddings in Transition-Based and Graph-Based Dependency Parsing – A Tale of Two Parsers Revisited \url{https://aclanthology.org/D19-1277v1.pdf}
    \item A structural probe for finding syntax in word representations. Manning et al. 2019
\end{itemize}
\end{frame}
