\section{Universal Dependencies Labels} \label{ud_labels}
\begin{supertabular}{| p{.20\textwidth} | p{.80\textwidth} |}
            \hline
            \textbf{Label} & \textbf{Meaning} \\
            \hline
            acl & Clausal Modifier Of Noun (Adnominal Clause) \\
            \hline
            advcl & Adverbial Clause Modifier \\
            \hline
            advmod & Adverbial Modifier \\
            \hline
            amod & Adjectival Modifier \\
            \hline
            appos & Appositional Modifier \\
            \hline
            aux & Auxiliary \\
            \hline
            case & Case Marking \\
            \hline
            cc & Coordinating Conjunction \\
            \hline
            ccomp & Clausal Complement \\
            \hline
            clf & Classifier \\
            \hline
            compound & Compound \\
            \hline
            conj & Conjunct \\
            \hline
            cop & Copula \\
            \hline
            csubj & Clausal Subject \\
            \hline
            dep & Unspecified Dependency \\
            \hline
            det & Determiner \\
            \hline
            discourse & Discourse Element \\
            \hline
            dislocated & Dislocated Elements \\
            \hline
            expl & Expletive \\
            \hline
            fixed & Fixed Multiword Expression \\
            \hline
            flat & Flat Multiword Expression \\
            \hline
            goeswith & Goes With \\
            \hline
            iobj & Indirect Object \\
            \hline
            list & List \\
            \hline
            mark & Marker \\
            \hline
            nmod & Nominal Modifier \\
            \hline
            nsubj & Nominal Subject \\
            \hline
            nummod & Numeric Modifier \\
            \hline
            obj & Object \\
            \hline
            obl & Oblique Nominal \\
            \hline
            orphan & Orphan \\
            \hline
            parataxis & Parataxis \\
            \hline
            punct & Punctuation \\
            \hline
            reparandum & Overridden Disfluency \\
            \hline
            root & Root \\
            \hline
            vocative & Vocative \\
            \hline
            xcomp & Open Clausal Complement \\
            \hline
        \caption{Arc labels and the meanings}
        \label{table:label_meanings}
\end{supertabular}

\section{Translation Information} \label{translation}
Some information about the human validator is tabulated below:
\begin{table}[ht]
    \begin{center}
        \begin{tabular}{|l|c|}
            \hline
            Age & 26 \\
            \hline
            Education & M.Sc. \\
            \hline
            Mother tongue & Nepali \\
            \hline
            Hindi proficiency & Fluent \\
            \hline
        \end{tabular}
        \caption{Validator Description}
        \label{table:validator_description}
    \end{center}
\end{table}

The following are all the errors reported by the validator: \\
\dev{ महानियंत्रक -> नियन्त्रक }\\
\dev{ लेखा -> महालेखा }\\
\dev{ पर -> को }\\
\dev{ उपाध्यक्ष -> उपसभामुख }\\
\dev{ किया -> गरेको }\\
\dev{ सत्ता -> अख्तियार }\\
\dev{ 19,401 -> 233 }\\
\dev{ आदिवासी -> जनजाति }\\
\dev{ उनकी -> आफ्नो }\\
\dev{ से -> ले }\\
\dev{ पश्चिमी -> पश्चिमाञ्चल }\\
\dev{ वह -> उनी }\\
\dev{ के -> को }\\
\dev{ आया -> आए }\\
\dev{ तरह -> तरिक }\\
\dev{ जीडीपी -> उत्पादन }\\
\dev{ २६.४ -> 26 }\\
\dev{ उपाध्यक्ष -> उपराष्ट्रपति }\\
\dev{ राजद -> आरजेडी }\\
\dev{ अध्यक्ष -> राष्ट्रपति }\\
\dev{ सामने -> नजिक }\\
\dev{ तो -> त्यसोभए }\\
\dev{ वाली -> वल }\\
\dev{ कंपनियां -> प्रदायकहरू }\\
\dev{ इस -> यसो }\\
\dev{ प्रचार -> प्रवर्द्धन }\\
\dev{ प्राधिकरण -> अख्तियार }\\
\dev{ रुख -> अडान }\\
\dev{ मानना -> मन्नु }\\
\dev{ मिले -> पाउने }\\
\dev{ अलावा -> बहेक }\\
\dev{ किया -> गरेको }\\
\dev{ वाले -> वल }\\
\dev{ वाला -> वल }\\
\dev{ है -> छैन }\\
\dev{ रेलवे -> रेल }\\
\dev{ के -> का }\\
\dev{ आहुति -> बलिदान }\\
\dev{ मातृकाएँ -> मातृ }\\
\dev{ पहले -> अगदि }\\

