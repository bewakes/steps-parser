\chapter{Annotated sentences samples} \label{dataset_summary_sample_appendix}
The following are 100 out of 200 annotated sentences created. Note that the
underscores(\_) represents the dummy token for the tokens that could not be perfectly aligned. For example:
\begin{itemize}
    \item For verbs like: \dev{करता है} whose translation is \dev{गर्दछ \_}
    \item For adjective phrases like: \dev{बड़ा सा} whose translation is \dev{ठुलो \_}
\end{itemize}


The following consist of a Hindi sentence, translated Nepali sentence and the
corresponding annotated data. And the tabulation of annotations is shown in
Table \ref{table:sample_annotation1}\\
\textbf{Hindi:} \dev{कुशीनगर का महत्‍व महापरिनिर्वाण मंदिर से है ।}\\
\textbf{Nepali:} \dev{कुशीनगर को महत्व महापरिनिर्वाण मन्दिर बाट हो ।}\\
\begin{table}[ht]
    \begin{center}
        \scalebox{0.8}{
            \begin{tabular}{|c|c|c|c|c|c|c|c|}
            \hline
            \textbf{\tiny{Index}} & \textbf{\tiny{Token}} & \textbf{\tiny{Lemma}} & \textbf{\tiny{UPOS}} & \textbf{\tiny{XPOS}} & \textbf{\tiny{Feats}} & \textbf{\tiny{Head}} & \textbf{\tiny{Label}} \\
            \hline
            1 & \scriptsize{\dev{कुशीनगर}} & \scriptsize{\dev{कुशीनगर}} & \scriptsize{PROPN} & \scriptsize{NNP} & \tiny{Case=Acc|Gender=Masc|Number=Sing|Person=3} & 3 & \scriptsize{nmod} \\
            \hline
            2 & \scriptsize{\dev{को}} & \scriptsize{\dev{का}} & \scriptsize{ADP} & \scriptsize{PSP} & \tiny{AdpType=Post|Case=Nom|Gender=Masc|Number=Sing} & 1 & \scriptsize{case} \\
            \hline
            3 & \scriptsize{\dev{महत्व}} & \scriptsize{\dev{महत्व}} & \scriptsize{NOUN} & \scriptsize{NN} & \tiny{Case=Nom|Gender=Masc|Number=Sing|Person=3} & 5 & \scriptsize{nsubj} \\
            \hline
            4 & \scriptsize{\dev{महापरिनिर्वाण}} & \scriptsize{\dev{महापरिनिर्वाण}} & \scriptsize{PROPN} & \scriptsize{NNPC} & \tiny{Case=Nom|Gender=Masc|Number=Sing|Person=3} & 5 & \scriptsize{compound} \\
            \hline
            5 & \scriptsize{\dev{मन्दिर}} & \scriptsize{\dev{मंदिर}} & \scriptsize{PROPN} & \scriptsize{NNP} & \tiny{Case=Acc|Gender=Masc|Number=Sing|Person=3} & 0 & \scriptsize{root} \\
            \hline
            6 & \scriptsize{\dev{बाट}} & \scriptsize{\dev{से}} & \scriptsize{ADP} & \scriptsize{PSP} & \tiny{AdpType=Post} & 5 & \scriptsize{case} \\
            \hline
            7 & \scriptsize{\dev{हो}} & \scriptsize{\dev{है}} & \scriptsize{AUX} & \scriptsize{VM} & \tiny{Mood=Ind|Number=Sing|Person=3|Tense=Pres|VerbForm=Fin} & 5 & \scriptsize{cop} \\
            \hline
            8 & \scriptsize{\dev{।}} & \scriptsize{\dev{।}} & \scriptsize{PUNCT} & \scriptsize{SYM} & \tiny{\_} & 5 & \scriptsize{punct} \\
            \hline
        \end{tabular}
        }
        \caption{Annotation for Sample sentence 1}
        \label{table:sample_annotation1}
    \end{center}
\end{table}


\subsubsection{Sentences}
Some of the translated nepali sentences are:
\begin{itemize}
    \item[1.] \dev{रामायण काल मा भगवान राम का छोरा कुश को राजधानी कुशावती लाई 483 ईसा पूर्व बुद्ध ले आफ्नो अन्तिम विश्राम को लागि चुने ।}

    \item[2.] \dev{मल्लहरू को राजधानी भए को कारण प्राचीनकाल मा यस स्थान को ठूलो महत्व थियो ।}

    \item[3.]\dev{हिन्दू राजाहरू को काल मा चीन बाट ह्युएन त्साङ , फाहियन र इट्सिङ ले आफ्नो यात्रा विवरण मा यस ठाउँ को महिमा को वर्णन गरेका छन् ।}

    \item[4.]\dev{कुशीनगर को सबैभन्दा ठूलो महत्त्व बौद्ध तीर्थ का रूप मा छ ।}

    \item[5.]\dev{कुशीनगर को सीमा मा प्रवेश गर्दा बित्तिकै भव्य प्रवेशद्वार तपाईंको स्वागत गर्दछ \_ ।}

    \item[6.]\dev{कुशीनगर को महत्व महापरिनिर्वाण मन्दिर बाट हो ।}

    \item[7.]\dev{यस मन्दिर को वास्तुकला अजन्ता को गुफा बाट प्रेरित छ ।}

    \item[8.]\dev{यो मन्दिर त्यही ठाउँ मा निर्माण गरिएको हो , जहाँ बाट यो मूर्ति निकालिएको \_ थियो ।}

    \item[9.]\dev{मन्दिर को पूर्व भाग मा एउटा स्तूप छ ।}

    \item[10.]\dev{यहाँ \_ भगवान बुद्ध को अंतिम दाहसंस्कार गरिएको \_ थियो ।}

    \item[11.]\dev{मूर्ति पनि अजन्ता को भगवान बुद्ध को महापरिनिर्वाण मूर्ति को प्रतिकृति हो ।}

    \item[12.]\dev{वैसे मूर्ति को अवधि अजन्ता भन्दा पहिले को हो ।}

    \item[13.]\dev{यस मन्दिर को वरिपरि धेरै विहारहरू ( जहाँ बौद्ध भिक्षुहरू बस्ने गर्दथे \_ ) र चैत्यहरू ( जहाँ भिक्षुहरू पूजा गर्थे \_ वा ध्यान लगाउँथे \_ ) भग्नावशेषहरू र खंडहर रहेका छन् जुन अशोककालीन भन्ने गरिन्छ \_ ।}
\end{itemize}

