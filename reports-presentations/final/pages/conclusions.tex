\chapter{Conclusions and Future Work}
This project shows an experimentation with rule-based system for dependency
parsing based on Paninian framework. The following are the conclusions:
\begin{itemize}
    \item LAS of 0.542 and UAS of 0.628 have been achieved against the dataset of 62 sentences tagged manually.
    \item Due to lack of dependency dataset, corpus based methods are not yet possible.
    \item Natural language, especially Nepali, has many constructs and thus requires lots of rules to address different language forms.
    \item The parser is still to handle quotations, conjunctions and reported speeches.
    \item The parser can also be extended to create knowledge graph and information retrieval systems.
\end{itemize}

\subsubsection{Future Work/Scope for Thesis}
This project can be extended in the following ways for thesis:
\begin{itemize}
    \item[1.] Add more rules and handle more complex sentences.
    \item[2.] Create a larger dataset in a semi-automated way - using the results from the parser and modifying where needed.
    \item[3.] Use the state-of-the-art BiLSTM models for dependency parsing using the created dataset.
\end{itemize}


