\chapter{Tagged Sentences}
> \dev{यसखण्ड अन्तर्गत म्याग्दी क्षेत्रमा पर्ने सडकको काम कुनैपनि सन्तोषजनक छैनन् ।} \\
> \dev{उसले गरेको यो प्रगतिमा गत वर्षातमा बाढीपहिरोले क्षति पुर्‍याएको छ ।} \\
> \dev{यस खण्डमा अहिले पूर्ण रुपमा काम बन्द छ ।} \\
> \dev{नेपाल राष्ट्र बैंकले पनि यस विषयमा चासो दिएको छ ।} \\
> \dev{विस्तारै गाउँमै रहने जनसङ्ख्या घट्दै गयो ।} \\
> \dev{फलतः भोलि पर्यावरण र खाद्य सन्तुलन दुवै क्षेत्रमा असर पुग्ने सम्भावना छ ।} \\
> \dev{बाँकी ७० प्रतिशत परिवारको बाँझो जग्गाको मात्रामा केही थपघट भएको छैन ।} \\
> \dev{समस्या यी तीन जिल्लाको मात्रै होइन ।} \\
> \dev{यो सालदेखि धान पनि लगाउँदिन ।} \\
> \dev{बुढापाका मात्र घर हुन्छन् ।} \\
> \dev{यो एकदमै ठूलो समस्या हो ।} \\
> \dev{अब हामी त्यतातिर लाग्छौँ ।’} \\
> \dev{त्यसपछि जमीन बाँझो राख्ने प्रवृत्ति निरन्तर बढ्दो छ ।} \\
> \dev{तर, यसका लागि उनी सरकारलाई नै दोष दिन्छन् ।} \\
> \dev{त्यो नभएको कारणले आजको अवस्था आएको हो ।’} \\
> \dev{जमीनलाई पैतृक सम्पत्ति मानेर जग्गा टुक्राटुक्रा बनाइन्छ ।} \\
> \dev{कृषि जैविक विविधता संरक्षण केन्द्रका प्रमुख श्रेष्ठ पनि भयावह तस्वीर प्रस्तुत गर्छन् ।} \\
> \dev{कोभिडले नै झन्डै–झन्डै फुटाएको थियो ।’} \\
> \dev{भौतिक रूपमा पनि भूक्षय बढ्दै जान्छ ।’} \\
> \dev{अझ अचेल त स्थानीय सरकारसँग पनि कृषि विकासका लागि बजेट हुन्छ ।} \\
> \dev{पण्डितको संयोजनमा तनहुँमा सञ्चालन भइरहेको परियोजना हाल रतनपुर बाहिर पनि फैलिइसकेको छ ।} \\
> \dev{त्यसै अनुसार हामीले नियमावलीमा यस्तो प्रस्ताव गरेका छौँ ।’} \\
> \dev{मूल समस्या नै यही हो ।’} \\
> \dev{सँगै थिए भरत दहाल पनि ।} \\
> \dev{मनमा उमङ्गको बहार थियो ।} \\
> \dev{छ घण्टाभन्दा लामो यात्रापछि उदयपुरको घुर्मी पुगियो ।} \\
> \dev{उदयपुर र ओखलढुङ्गाको सीमाना हर्कपुरमा चिया पियौँ ।} \\
> \dev{घरबाट दुईजना निस्किएका हामी त्यहाँ पुग्दा चारजना भइसकेका थियौं ।} \\
> \dev{हर्कपुरमा पहिलोपटक चियामात्र पिइएन ।} \\
> \dev{राति अबेला ओखलढुङ्गामा बास बस्न पुगेका थियौँ ।} \\
> \dev{त्यसो त ओखलढुङ्गा निकै सृजनात्मक भूमि हो ।} \\
> \dev{सल्लेरीमा दिनभरको घुमफिरपछि साँझ ओखलढुङ्गा नै झर्ने योजना थियो ।} \\
> \dev{एकटक हिमाललाई आँखामा कैद गरेपछि त्यहाँबाट फिर्ने योजना थियो ।} \\
> \dev{साँझमा सल्लेरीबाट हिँड्यौँ ।} \\
> \dev{निकैबेरको यात्रापछि हामी ओखलढुङ्गा आइपुग्यौँ ।} \\
> \dev{दिक्तेल जाँदा बाटामा रमाइला दृश्यहरू आँखा र क्यामेरामा कैद गर्यौँ ।} \\
> \dev{साँझमा दिक्तेलको बसाइ रोचक भयो ।} \\
> \dev{बुईपामा आएर एकफेर कवि देवान किराँतीलाई फोन गरेँ ।} \\
> \dev{देवान उनै जसराज किराँतीका सुपुत्र पनि हुन् ।} \\
> \dev{निकैबेरको यात्रापछि रसुवा पुगिसकेका थियौँ ।} \\
> \dev{अहो ! कहिलेकाहीँ कस्तो संकटमा परिन्छ ।} \\
> \dev{दूरबाट एउटा बस आउँदै गरेको देखेँ ।} \\
> \dev{मावलबाट सोही बसमा रसुवासम्म आउने कार्यक्रम थियो ।} \\
> \dev{रसुवामा बनाइसकेपछि हामी गाइघाटतर्फ हिँड्यौँ ।} \\
> \dev{त्यही साँझ घर पुग्ने योजना सफल हुँदै थियो ।} \\
> \dev{केहीबेरको यात्रामै पुनः मोटरसाइकल बिग्रियो ।} \\
> \dev{एकैठाउँमा पाँच घण्टाको पर्खाइ साह्रै अत्यासलाग्दो थियो ।} \\
> \dev{पाँच बजे रसुवाबाट गाडी आउनेरहेछ ।} \\
> \dev{जसले प्यास मेट्ने पानी समेत दिन सकेन ।} \\
> \dev{अब त सधैंसधैंका लागि साथी मबाट बिदा भएको छ ।} \\
> \dev{अहिले हाम्रा उत्पादन हरू मा अस्बेस्टस छैन ।} \\
> \dev{कारखाना का क्रोसिडोलाइट प्रयोग गरिएका क्षेत्र हरू विशेष रूप ले धुलाम्मे थिए ।} \\
> \dev{यस को आज हाम्रो श्रमसमुह सँग कुनै सरोकार छैन ।} \\
> \dev{उनी २० वर्ष सम्म क्रिस्लर को बिक्री तथा बजार कार्यकारी रहेका थिए ।} \\
> \dev{यो वर्ष होइन ।} \\
> \dev{यस पछि स्याम्पेन र डेजर्ट सुरु भयो ।} \\
> \dev{नयाँ सेयर को लागि मूल्य निर्धारण गरिएको छैन ।} \\
> \dev{बरू , यो निर्णय गर्न कम्पनी ले बजार लाई छोड्नेछ ।} \\
> \dev{क्रे कम्प्युटर ले नास्दाक मा कारोबार गर्न आवेदन दिएको छ ।} \\
> \dev{सबै क्रे रिसर्च बाट आएका हुन् ।} \\
> \dev{पहिला उहाँ इस्टर्न एडिसन को उपाध्यक्ष हुनुहुन्थ्यो ।} \\
> \dev{उहाँ पहिला उपाध्यक्ष हुनुहुन्थ्यो ।} \\
