\section{Experiments and training times} \label{experiments_time_appendix}

The above experiments were run for 100 epochs on training sets containing
around 12000 to 15000 sentences, except for random training on the first row
and the ones mentioning sample size.

\subsubsection{Experiments without language embeddings}
\begin{table}[ht]
    \begin{center}
        \scalebox{0.9}{
        \begin{tabular}{|c|c|c|c|}
            \hline
            \textbf{Exp. code} & \textbf{Datasets} & \textbf{Embs?} & \textbf{Time(hrs)} \\
            \hline
            rand & Random weights & No & 0.2 \\
            \hline
            en2k & English 2k samples & No & 3 \\
            \hline
            en5k & English 5k samples & No & 5 \\
            \hline
            en & English & No & 12 \\
            \hline
            hi & Hindi & No & 15 \\
            \hline
            en\_hi & English, Hindi & No & 18 \\
            \hline
            multi & multi & No & 18 \\
            \hline
            multi\_np20 & \textbf{np},multi & No & 14 \\
            \hline
            \textbf{multi\_np100} & \textbf{np},multi & No & 14 \\
            \hline
            hi\_np20 & \textbf{np},multi & No & 12 \\
            \hline
            \textbf{hi\_np100} & \textbf{np},multi & No & 12.5 \\
            \hline
            \textbf{np100} & \textbf{np} & No & 0.4 \\
            \hline
        \end{tabular}
        }
        \caption{Training duration without language embeddings}
        \label{table:experiments_durations_noem}
    \end{center}
\end{table}

\subsubsection{Experiments with language embeddings}
\begin{table}[ht]
    \begin{center}
        \scalebox{0.85}{
        \begin{tabular}{|c|c|c|c|}
            \hline
            \textbf{Exp. code} & \textbf{Datasets} & \textbf{Embs?} & \textbf{Time(hrs)} \\
            \hline
            en2kem & English 2k samples & Yes & 5 \\
            \hline
            en5kem & English 5k samples & Yes & 7 \\
            \hline
            enem & English & Yes & 15 \\
            \hline
            hiem & Hindi & Yes & 18 \\
            \hline
            en\_hi\_em & English, Hindi & Yes & 21 \\
            \hline
            multiem & multi & Yes & 21 \\
            \hline
            \textbf{multi\_np20\_em} & \textbf{np},multi & Yes & 16 \\
            \hline
        \end{tabular}
    }
        \caption{Training duration with language embeddings}
        \label{table:experiments_durations_em}
    \end{center}
\end{table}
The training times for \textit{en\_hi} and \textit{multi} are slightly higher
because for \textit{en\_hi} almost all of the data(10000 sentences for English
and 12000 for Hindi) were used. Similarly, for \textit{multi} setup, 1500
sentences for each language were used. For other multi setups only 1000
sentences from source languages were used for training.

\subsubsection{Experiments time summary}
\begin{table}[ht]
    \begin{center}
        \begin{tabular}{|l|c|}
            \hline
            \textbf{Total Training duration} & ~228 hrs \\
            \hline
            \textbf{Average Training duration} & ~13 hrs \\
            \hline
            \textbf{Total experiments} & 19 \\
            \hline
        \end{tabular}
    \end{center}
\end{table}

\section{Annotated sentences samples} \label{dataset_summary_sample_appendix}

The following are 100 out of 200 annotated sentences created. Note that the
underscores(\_) represents the dummy token for the tokens that could not be perfectly aligned. For example:
\begin{itemize}
    \item For verbs like: \dev{करता है} whose translation is \dev{गर्दछ \_}
    \item For adjective phrases like: \dev{बड़ा सा} whose translation is \dev{ठुलो \_}
\end{itemize}


The following consist of a Hindi sentence, translated Nepali sentence and the
corresponding annotated data. And the tabulation of annotations is shown in
Table \ref{table:sample_annotation1}\\
\textbf{Hindi:} \dev{कुशीनगर का महत्‍व महापरिनिर्वाण मंदिर से है ।}\\
\textbf{Nepali:} \dev{कुशीनगर को महत्व महापरिनिर्वाण मन्दिर बाट हो ।}\\
\begin{table}[ht]
    \begin{center}
        \scalebox{0.82}{
        \begin{tabular}{|c|c|c|c|c|c|c|}
            \hline
            \textbf{\tiny{Index}} & \textbf{\tiny{Token}} & \textbf{\tiny{Lemma}} & \textbf{\tiny{UPOS}} & \textbf{\tiny{XPOS}} & \textbf{\tiny{Head}} & \textbf{\tiny{Label}} \\
            \hline
            1 & \scriptsize{\dev{कुशीनगर}} & \scriptsize{\dev{कुशीनगर}} & \scriptsize{PROPN} & \scriptsize{NNP} & 3 & \scriptsize{nmod} \\
            \hline
            2 & \scriptsize{\dev{को}} & \scriptsize{\dev{का}} & \scriptsize{ADP} & \scriptsize{PSP} & 1 & \scriptsize{case} \\
            \hline
            3 & \scriptsize{\dev{महत्व}} & \scriptsize{\dev{महत्व}} & \scriptsize{NOUN} & \scriptsize{NN} & 5 & \scriptsize{nsubj} \\
            \hline
            4 & \scriptsize{\dev{महापरिनिर्वाण}} & \scriptsize{\dev{महापरिनिर्वाण}} & \scriptsize{PROPN} & \scriptsize{NNPC} & 5 & \scriptsize{compound} \\
            \hline
            5 & \scriptsize{\dev{मन्दिर}} & \scriptsize{\dev{मंदिर}} & \scriptsize{PROPN} & \scriptsize{NNP} & 0 & \scriptsize{root} \\
            \hline
            6 & \scriptsize{\dev{बाट}} & \scriptsize{\dev{से}} & \scriptsize{ADP} & \scriptsize{PSP} & 5 & \scriptsize{case} \\
            \hline
            7 & \scriptsize{\dev{हो}} & \scriptsize{\dev{है}} & \scriptsize{AUX} & \scriptsize{VM} & 5 & \scriptsize{cop} \\
            \hline
            8 & \scriptsize{\dev{।}} & \scriptsize{\dev{।}} & \scriptsize{PUNCT} & \scriptsize{SYM} & 5 & \scriptsize{punct} \\
            \hline
        \end{tabular}
        }
        \caption{Annotation for Sample sentence 1}
        \label{table:sample_annotation1}
    \end{center}
\end{table}


\subsubsection{Sentences}
Some of the translated nepali sentences are:
\begin{itemize}
    \item[1.] \dev{रामायण काल मा भगवान राम का छोरा कुश को राजधानी कुशावती लाई 483 ईसा पूर्व बुद्ध ले आफ्नो अन्तिम विश्राम को लागि चुने ।}

    \item[2.] \dev{मल्लहरू को राजधानी भए को कारण प्राचीनकाल मा यस स्थान को ठूलो महत्व थियो ।}

    \item[3.]\dev{हिन्दू राजाहरू को काल मा चीन बाट ह्युएन त्साङ , फाहियन र इट्सिङ ले आफ्नो यात्रा विवरण मा यस ठाउँ को महिमा को वर्णन गरेका छन् ।}

    \item[4.]\dev{कुशीनगर को सबैभन्दा ठूलो महत्त्व बौद्ध तीर्थ का रूप मा छ ।}

    \item[5.]\dev{कुशीनगर को सीमा मा प्रवेश गर्दा बित्तिकै भव्य प्रवेशद्वार तपाईंको स्वागत गर्दछ \_ ।}

    \item[6.]\dev{कुशीनगर को महत्व महापरिनिर्वाण मन्दिर बाट हो ।}

    \item[7.]\dev{यस मन्दिर को वास्तुकला अजन्ता को गुफा बाट प्रेरित छ ।}

    \item[8.]\dev{यो मन्दिर त्यही ठाउँ मा निर्माण गरिएको हो , जहाँ बाट यो मूर्ति निकालिएको \_ थियो ।}

    \item[9.]\dev{मन्दिर को पूर्व भाग मा एउटा स्तूप छ ।}

    \item[10.]\dev{यहाँ \_ भगवान बुद्ध को अंतिम दाहसंस्कार गरिएको \_ थियो ।}

    \item[11.]\dev{मूर्ति पनि अजन्ता को भगवान बुद्ध को महापरिनिर्वाण मूर्ति को प्रतिकृति हो ।}

    \item[12.]\dev{वैसे मूर्ति को अवधि अजन्ता भन्दा पहिले को हो ।}

    \item[13.]\dev{यस मन्दिर को वरिपरि धेरै विहारहरू ( जहाँ बौद्ध भिक्षुहरू बस्ने गर्दथे \_ ) र चैत्यहरू ( जहाँ भिक्षुहरू पूजा गर्थे \_ वा ध्यान लगाउँथे \_ ) भग्नावशेषहरू र खंडहर रहेका छन् जुन अशोककालीन भन्ने गरिन्छ \_ ।}
\end{itemize}

\section{Universal Dependencies Labels} \label{ud_labels}
\begin{table}[ht]
    \begin{center}
        \scalebox{0.8}{
\begin{tabular}{|c|c|}
            \hline
            \textbf{Label} & \textbf{Meaning} \\
            \hline
            acl & Clausal Modifier Of Noun (Adnominal Clause) \\
            \hline
            advcl & Adverbial Clause Modifier \\
            \hline
            advmod & Adverbial Modifier \\
            \hline
            amod & Adjectival Modifier \\
            \hline
            appos & Appositional Modifier \\
            \hline
            aux & Auxiliary \\
            \hline
            case & Case Marking \\
            \hline
            cc & Coordinating Conjunction \\
            \hline
            ccomp & Clausal Complement \\
            \hline
            clf & Classifier \\
            \hline
            compound & Compound \\
            \hline
            conj & Conjunct \\
            \hline
            cop & Copula \\
            \hline
            csubj & Clausal Subject \\
            \hline
            dep & Unspecified Dependency \\
            \hline
            det & Determiner \\
            \hline
            discourse & Discourse Element \\
            \hline
            dislocated & Dislocated Elements \\
            \hline
            expl & Expletive \\
            \hline
            fixed & Fixed Multiword Expression \\
            \hline
            flat & Flat Multiword Expression \\
            \hline
            goeswith & Goes With \\
            \hline
            iobj & Indirect Object \\
            \hline
            list & List \\
            \hline
            mark & Marker \\
            \hline
            nmod & Nominal Modifier \\
            \hline
            nsubj & Nominal Subject \\
            \hline
            nummod & Numeric Modifier \\
            \hline
            obj & Object \\
            \hline
            obl & Oblique Nominal \\
            \hline
            orphan & Orphan \\
            \hline
            parataxis & Parataxis \\
            \hline
            punct & Punctuation \\
            \hline
            reparandum & Overridden Disfluency \\
            \hline
            root & Root \\
            \hline
            vocative & Vocative \\
            \hline
            xcomp & Open Clausal Complement \\
            \hline
        \end{tabular}
        }
        \caption{Arc labels and the meanings}
        \label{table:label_meanings}
    \end{center}
\end{table}

\section{Translation Information} \label{translation}
Some information about the human validator is tabulated below:
\begin{table}[ht]
    \begin{center}
        \begin{tabular}{|l|c|}
            \hline
            Age & 26 \\
            \hline
            Education & M.Sc. \\
            \hline
            Mother tongue & Nepali \\
            \hline
            Hindi proficiency & Fluent \\
            \hline
        \end{tabular}
        \caption{Validator Description}
        \label{table:validator_description}
    \end{center}
\end{table}

The following are all the errors reported by the validator: \\
\dev{ महानियंत्रक -> नियन्त्रक }\\
\dev{ लेखा -> महालेखा }\\
\dev{ पर -> को }\\
\dev{ उपाध्यक्ष -> उपसभामुख }\\
\dev{ किया -> गरेको }\\
\dev{ सत्ता -> अख्तियार }\\
\dev{ 19,401 -> 233 }\\
\dev{ आदिवासी -> जनजाति }\\
\dev{ उनकी -> आफ्नो }\\
\dev{ से -> ले }\\
\dev{ पश्चिमी -> पश्चिमाञ्चल }\\
\dev{ वह -> उनी }\\
\dev{ के -> को }\\
\dev{ आया -> आए }\\
\dev{ तरह -> तरिक }\\
\dev{ जीडीपी -> उत्पादन }\\
\dev{ २६.४ -> 26 }\\
\dev{ उपाध्यक्ष -> उपराष्ट्रपति }\\
\dev{ राजद -> आरजेडी }\\
\dev{ अध्यक्ष -> राष्ट्रपति }\\
\dev{ सामने -> नजिक }\\
\dev{ तो -> त्यसोभए }\\
\dev{ वाली -> वल }\\
\dev{ कंपनियां -> प्रदायकहरू }\\
\dev{ इस -> यसो }\\
\dev{ प्रचार -> प्रवर्द्धन }\\
\dev{ प्राधिकरण -> अख्तियार }\\
\dev{ रुख -> अडान }\\
\dev{ मानना -> मन्नु }\\
\dev{ मिले -> पाउने }\\
\dev{ अलावा -> बहेक }\\
\dev{ किया -> गरेको }\\
\dev{ वाले -> वल }\\
\dev{ वाला -> वल }\\
\dev{ है -> छैन }\\
\dev{ रेलवे -> रेल }\\
\dev{ के -> का }\\
\dev{ आहुति -> बलिदान }\\
\dev{ मातृकाएँ -> मातृ }\\
\dev{ पहले -> अगदि }\\

