\chapter{Results}
Basic parsing of simple sentences have been carried out. Benchmarking dataset
is also created against which the metrics are evaluated.

\section{Sample Output}
\subsubsection{Sentence 1}
\dev{मोटो रामले खाना खाएर मोहनलाई झ्यालको किताब हातले दियो ।}
\\~\\
Pre-processed and POS-Tagged result:
\\~\\
(\dev{मोटो}, JJM) (\dev{राम}, NNP) (\dev{ले}, PLE) (\dev{खाना}, NN) (\dev{खाएर}, VBO)
(\dev{मोहन}, NNP) (\dev{लाई}, PLAI) (\dev{झ्याल}, NN) (\dev{को}, PKO) (\dev{किताब},
NN) (\dev{हात}, NN) (\dev{ले}, PLE) (\dev{दियो}, VBF) (\dev{।}, YF)
\\~\\
Initial chunking result:
\\~\\
\begin{table}[h]
\begin{center}
\begin{tabular}{|c|c|c|}
\hline
    \textbf{Chunk} & \textbf{Possible karak roles} & \textbf{CSP Variables} \\
    \hline 
(\dev{मोटो}, JJM) & [adjective] & \code{adjective\_0} \\ 
\hline 
(\dev{राम}, NNP) (\dev{ले}, PLE) & [karta, karan] & \code{karta\_1, karan\_1} \\ 
\hline 
(\dev{खाना}, NN) (\dev{खाएर}, VBO) & [vmod] & \code{vmod\_2} \\ 
\hline 
(\dev{मोहन}, NNP) (\dev{लाई}, PLAI) & [karma, sampradan] & \code{karma\_3, sampradan\_3} \\ 
\hline 
(\dev{झ्याल}, NN) (\dev{को}, PKO) & [sambandha] & \code{sambandha\_4} \\ 
\hline 
(\dev{किताब}, NN) & [karta, karma] & \code{karta\_5, karma\_5} \\ 
\hline 
(\dev{हात}, NN) (\dev{ले}, PLE) & [karta, karan] & \code{karta\_6, karan\_6} \\ 
\hline 
(\dev{दियो}, VBF) & [kriya] & \code{kriya\_7} \\ 
\hline 

\end{tabular}
\end{center}
\end{table}
\\~\\
Setting up constraints:
\\
\code{SEED}: (\dev{दिनु} \code{past}) \\
\code{
mitting non-LE karta for sakarmak kriya: karta\_5 == 0\\
SETTING mandatory karta: sum ['karta\_1', 'karta\_5', 'karta\_6'] == 1 \\
SETTING OPTIONAL karan: sum ['karan\_1', 'karan\_6']  <= 1 \\
SETTING OPTIONAL karma: sum ['karma\_3', 'karma\_5']  <= 1 \\
SETTING OPTIONAL sampradan: sum ['sampradan\_3']  <= 1 \\
Cooccuring  karta karan : karta\_1 - karan\_1 > 0\\
SETTING no karan before karta: karta\_1 - karan\_1 >= 0
SETTING single karta constraint: sum ['karta\_1', 'karta\_5', 'karta\_6']  <= 1\\
SETTING single karma constraint: sum ['karma\_3', 'karma\_5'] <= 1\\
}
Final constraint solved result:
\\~\\
\begin{figure}[h]
    \center
    \includegraphics[scale=0.8]{midterm_result_1}
    \caption{Dependency parse of sentence 1}
    \label{fig:result_1}
\end{figure}

\subsubsection{Sentence 2}
\dev{अहिले हाम्रा उत्पादन हरू मा अस्बेस्टस छैन ।}
\\~\\
Pre-processed and POS-Tagged result:
\\~\\
(\dev{अहिले}, RBO) (\dev{हाम्रा}, PP\$) (\dev{उत्पादन}, NN) (\dev{हरू}, HRU) (\dev{मा}, POP) (\dev{अस्बेस्टस}, NNP) (\dev{छैन}, VBX) (\dev{।}, YF)
\\~\\
Initial chunking result:
\\
\begin{table}[h]
\begin{center}
\begin{tabular}{|c|c|c|}
\hline
    \textbf{Chunk} & \textbf{Possible karak roles} & \textbf{CSP Variables} \\
    \hline
(\dev{अहिले}, RBO) & [vmod] & \code{vmod\_0} \\ 
\hline 
(\dev{हाम्रा}, PP\$) & [sambandha] & \code{sambandha\_1} \\ 
\hline 
(\dev{उत्पादन}, NN) (\dev{हरू}, HRU) (\dev{मा}, POP) & [adhikaran] & \code{adhikaran\_2} \\ 
\hline 
(\dev{अस्बेस्टस}, NNP) & [karta, karma] & \code{karta\_3, karma\_3} \\ 
\hline 
(\dev{छैन}, VBX) & [kriya] & \code{kriya\_4} \\ 
\hline 

\end{tabular}
\end{center}
\end{table}
\code{
SETTING OPTIONAL adhikaran: sum ['adhikaran\_2']  <= 1 \\
SETTING mandatory karta: sum ['karta\_3'] == 1 \\
SETTING OPTIONAL karma: sum ['karma\_3']  <= 1 \\
SETTING single karta constraint: sum ['karta\_3']  <= 1\\
SETTING single karma constraint: sum ['karma\_3'] <= 1\\
}
Final constraint solved result:
\\~\\
\begin{figure}[h]
    \center
    \includegraphics[scale=0.8]{midterm_result_2}
    \caption{Dependency parse of sentence 2}
    \label{fig:result_2}
\end{figure}

\subsubsection{Sentence 3}
\dev{बरू , यो निर्णय गर्न कम्पनी ले बजार लाई छोड्नेछ ।}
\\~\\
Pre-processed and POS-Tagged result:
\\~\\
(\dev{बरू}, CS) (\dev{,}, YM) (\dev{यो}, DUM) (\dev{निर्णय}, NN) (\dev{गर्न}, VBI) (\dev{कम्पनी}, NN) (\dev{ले}, PLE) (\dev{बजार}, NN) (\dev{लाई}, PLAI) (\dev{छोड्नेछ}, VBF)
\\~\\
Initial chunking result:
\\~\\
\begin{table}[h]
\begin{center}
\begin{tabular}{|c|c|c|}
\hline
    \textbf{Chunk} & \textbf{Possible karak roles} & \textbf{CSP Variables} \\
    \hline 
(\dev{बरू}, CS) & [unknown] & \code{unknown\_0} \\ 
\hline 
(\dev{,}, YM) & [unknown] & \code{unknown\_1} \\ 
\hline 
(\dev{यो}, DUM) & [adjective] & \code{adjective\_2} \\ 
\hline 
(\dev{निर्णय}, NN) (\dev{गर्न}, VBI) & [vmod] & \code{vmod\_3} \\ 
\hline 
(\dev{कम्पनी}, NN) (\dev{ले}, PLE) & [karta, karan] & \code{karta\_4, karan\_4} \\ 
\hline 
(\dev{बजार}, NN) (\dev{लाई}, PLAI) & [karma, sampradan] & \code{karma\_5, sampradan\_5} \\ 
\hline 
(\dev{छोड्नेछ}, VBF) & [kriya] & \code{kriya\_6} \\ 
\hline 

\end{tabular}
\end{center}
\end{table}
\\~\\
Setting up constraints:
\\
\code{SEED}: (\dev{छोड्नु} future ) \\
\code{
SETTING mandatory karta: sum ['karta\_4'] == 1 \\
SETTING OPTIONAL karan: sum ['karan\_4']  <= 1 \\
SETTING OPTIONAL karma: sum ['karma\_5']  <= 1 \\
SETTING OPTIONAL sampradan: sum ['sampradan\_5']  <= 1 \\
SETTING no karan before karta: karta\_4 - karan\_4 >= 0
SETTING single karta constraint: sum ['karta\_4']  <= 1\\
SETTING single karma constraint: sum ['karma\_5'] <= 1\\
}
Final constraint solved result:
\\~\\
\begin{figure}[h]
    \center
    \includegraphics[scale=0.8]{midterm_result_3}
    \caption{Dependency parse of sentence 3}
    \label{fig:result_3}
\end{figure}

\subsubsection{Sentence 4}
\dev{साउदर्न अप्टिकल को बिक्री कार्यक्रम को एउटा भाग हो ।}
\\~\\
Pre-processed and POS-Tagged result:
\\~\\
(\dev{साउदर्न}, NNP) (\dev{अप्टिकल}, NNP) (\dev{को}, PKO) (\dev{बिक्री}, NN) (\dev{कार्यक्रम}, NN) (\dev{को}, PKO) (\dev{एउटा}, CL) (\dev{भाग}, NN) (\dev{हो}, VBX) (\dev{।}, YF)
\\~\\
Initial chunking result:
\\~\\
\begin{table}
\begin{center}
\begin{tabular}{|c|c|c|}
\hline
    \textbf{Chunk} & \textbf{Possible karak roles} & \textbf{CSP Variables} \\
    \hline 
(\dev{साउदर्न}, NNP) (\dev{अप्टिकल}, NNP) (\dev{को}, PKO) & [sambandha] & \code{sambandha\_0} \\ 
\hline 
(\dev{बिक्री}, NN) & [karta, karma] & \code{karta\_1, karma\_1} \\ 
\hline 
(\dev{कार्यक्रम}, NN) (\dev{को}, PKO) & [sambandha] & \code{sambandha\_2} \\ 
\hline 
(\dev{एउटा}, CL) & [adjective] & \code{adjective\_3} \\ 
\hline 
(\dev{भाग}, NN) & [karta, karma] & \code{karta\_4, karma\_4} \\ 
\hline 
(\dev{हो}, VBX) & [kriya] & \code{kriya\_5} \\ 
\hline 
\end{tabular}
\end{center}
\end{table}
\\~\\
Setting up constraints:
\\
\code{
SETTING mandatory karta: sum ['karta\_1', 'karta\_4'] == 1 \\
SETTING OPTIONAL karma: sum ['karma\_1', 'karma\_4']  <= 1 \\
Cooccuring  karta karma : karta\_1 - karma\_1 > 0\\
SETTING single karta constraint: sum ['karta\_1', 'karta\_4']  <= 1\\
SETTING single karma constraint: sum ['karma\_1', 'karma\_4'] <= 1\\
}
Final constraint solved result:
\\~\\
\begin{figure}[h]
    \center
    \includegraphics[scale=0.8]{midterm_result_4}
    \caption{Dependency parse of sentence 4}
    \label{fig:result_4}
\end{figure}

\normalfont

\section{Summary}
The tagged dataset contains 62 sentences tagged using a tagging user interface.
The data is summarized below.
\begin{table}[h]
    \begin{center}
        \begin{tabular}{|c|c|}
            \hline
            \textbf{Data} & \textbf{Count} \\
            \hline
            Sentences & 62 \\
            \hline
            Words/Tokens & 512 \\
            \hline
            Attachments/Links & 387 \\
            \hline
        \end{tabular}
    \end{center}
    \caption{Data Summary}
    \label{table:data_summary}
\end{table}
Formulating as a Constraint Satisfaction Problem, a sentence is either parsed
or not at all because of the discrete nature of CSP variables. The LAS  and UAS are calculated for
only the sentences that had a solution \textendash { } about 50 sentences.
Chunking has not been evaluated as it will be indicated by LAS and UAS.
The results are tabulated below.
\begin{table}[h]
    \begin{center}
        \begin{tabular}{|c|c|}
            \hline
            \textbf{Metric} & \textbf{Accuracy} \\
            \hline
            LAS & 0.542 \\
            \hline
            UAS & 0.628 \\
            \hline
            POS Tagging & 0.96 \\
            \hline
            Solution Accuracy & 0.758 \\
            \hline
        \end{tabular}
    \end{center}
    \caption{Accuracy result}
    \label{table:accuracy_result}
\end{table}

\section{Error Analysis}
The POS Tagging accuracy seems to be excellent due to the use of the reliable
TnT Tagger. The LAS and UAS are not very exciting. Although close to 0.5, it is
important to note that these are still much better than random assignments.
This is because, a sentence of average length 10 has 9 links among the words
which has approximately ${}_{9}C_{2}$ possible combinations and each link can have about 10
labels. Thus, the probability for correctly finding and labeling links for a
random tagger is much less than 0.5.
\\~\\
The existing errors are mainly due to the following:
\begin{itemize}
    \item[a.] Presence of multi-word noun phrases. For example: \dev{मेनफ्रेम कम्प्युटरहरु}
        \normalfont
        is chunked as two nouns instead of single entity. When this
        chunking is used, the CSP can't find solutions because both of these
        terms are treated as \textit{karta}.
    \item[b.] Incomplete or imperfect chunk rules. There are still more chunk
        rules to be discovered and added. For example: a rule for
        \textit{sampradan} \textendash { } \dev{- को/का लागि}
        \normalfont
        is not yet incorporated.
        Similarly, chunking is not working well for phrases containing an
        adjective(JJ) and infinitive(VBI) e.g. \dev{उपस्थित हुन}
        \normalfont
    \item[c.] Some of the verb modifier chunks are also not in the rule. For
        example, \dev{२० वर्ष सम्म}
        \normalfont
   \itemsep0em
    \item[d.] Some verb identifiers are chunked incorrectly. For example,
        \dev{भने छैनन्}
        \normalfont
        is chunked as a verb phrase but in fact it consists of an
        adverb and verb. The rule set needs to include such constructs as well.
    \item[e.] The dataset consists of substantial amount of complex sentences
        like quotations and conjunctions. There are no rules for such sentences
        yet.
\end{itemize}
