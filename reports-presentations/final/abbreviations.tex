\addcontentsline{toc}{chapter}{Abbreviations} \noindent
%\renewcommand{\nomname}{List of Abbreviations}

%--- Acronyms -----------------------------------------------------------------%
% \acrodef{label}[acronym]{written out form} % acronym syntax
%\acrodef{etacar}[$\eta$ Car]{Eta Carinae}   % acronym example
%--- Acronyms -----------------------------------------------------------------%
% how to use acronyms:
% \ac = use acronym, first time write both, full name and acronym
% \acf = use full name (text + acronym)
% \acs = only use acronym
% \acl = only use long text
% \acp, acfp, acsp, aclp = use plural form for acronym (append 's')
% \acsu, aclu = write + mark as used
% \acfi = write full name in italics and acronym in normal style
% \acused = mark acronym as used
% \acfip = full, emphasized, plural, used
%--- Acronyms -----------------------------------------------------------------%

\chapter*{LIST OF ABBREVIATIONS}
\begin{acronym}
        \acro{las}[LAS]{Labeled Attachment Score}
        \acro{mbert}[mBERT]{Multilingual-Bidirectional Encoder Representations from Transformers}
        \acro{nlp}[NLP]{Natural Language Processing}
        \acro{pos}[POS]{Part of Speech}
        \acro{rnn}[RNN]{Recurrent Neural Network}
        \acro{uas}[UAS]{Unlabeled Attachment Score}
        \acro{ud}[UD]{Universal Dependencies}
        \acro{wcdg}[WCDG]{Weighted Constraint Dependency Grammar}
\end{acronym}
